\subsection{/user/request}

\begin{lstlisting}
    GET /user/request
\end{lstlisting}

\textbf{Beschreibung:} \\
Gibt alle Daten zurück, welche über den angemeldeten Nutzer im System gespeichert sind. Wird genutzt, um einen Datenexport nach DSGVO zu ermöglichen.

\textbf{Authorization-Typ:} \\
App

\subsubsection{Parameter}

Keine

\subsubsection{Antwort (bei Erfolg)}

\lstinline{Content-Type: application/json}
\begin{lstlisting}
    {
        "success": true, 
        "data": {
            "userData": {
                ...
            },
            "orderData": {
                ...
            },
            "sessionData": {
                ...
            },
            "verificationData": {
                ...
            }
        }
    }
\end{lstlisting}

Hierbei stellt \lstinline{userData} alle Nutzer-Metadaten wie E-Mail-Adresse oder Zeitpunkt der Registrierung dar. \lstinline{orderData} beinhaltet alle Daten über Bestellungen, welche dem Nutzer zugeordnet werden können. \lstinline{sessionData} alle Daten über Sessions, welche dem Nutzer zugeordnet werden können und \lstinline{verificationData} alle Daten über die Verifizierung des Nutzers (beispielsweise die generierte Verifizierungs-URL).

\subsubsection{Rate Limit}

Diese API-Route verfügt über ein Rate-Limit, um Missbrauch vorzubeugen. Es kann maximal eine Abfrage innerhalb von 24 Stunden von derselben IP-Adresse durchgeführt werden.