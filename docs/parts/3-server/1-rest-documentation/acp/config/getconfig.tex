\subsection{/acp/config/get}

\begin{lstlisting}
    GET /acp/config/get
\end{lstlisting}

\textbf{Beschreibung:} \\
Gibt alle im System gespeicherten Config-Einträge zurück. Wird genutzt, um eine Auflistung der Config-Einträge im ACP zu ermöglichen.

\textbf{Authorization-Typ:} \\
ACP

\subsubsection{Parameter}
Keine

\subsubsection{Antwort (bei Erfolg)}

\lstinline{Content-Type: application/json}
\begin{lstlisting}
    {
        "success": true, 
        "configEntries": [
            {
                "key": "closingTime",
                "friendlyName": "Closing Time",
                "type": "TIME",
                "value": "20:00"
            }
        ]
    }
\end{lstlisting}

Wobei \lstinline{type} ein Enum aus den Werten \lstinline{INTEGER}, \lstinline{STRING} und \lstinline{TIME} ist.