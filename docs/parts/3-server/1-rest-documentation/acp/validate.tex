\subsection{/acp/validate}

\begin{lstlisting}
    POST /acp/validate
\end{lstlisting}

\textbf{Beschreibung:} \\
Prüft, ob der Session-Token eines ACP-Nutzerkontos gültig ist. Wird genutzt, um bei Aufruf des ACPs einen vorhandenen Session-Cookie zu prüfen.

\textbf{Authorization-Typ:} \\
Keine

\subsubsection{Payload}

\lstinline{name: string} (benötigt) \\
$\rightarrow$ Der Name des ACP-Nutzerkontos, bei dem die Validierung durchgeführt werden soll
\lstinline{token: string} (benötigt) \\
$\rightarrow$ Der Session-Token des ACP-Nutzerkontos, bei dem die Validierung durchgeführt werden soll

\subsubsection{Beispiel}

\begin{lstlisting}
    POST /acp/validate
    {
        "name": "admin",
        "token": "7sGCYBi5796mvPYe1Jg5pybMuJXBGa%2B4"
    }
\end{lstlisting}

Validiert den Session-Token \lstinline{7sGCYBi5796mvPYe1Jg5pybMuJXBGa%2B4} für das ACP-Nutzerkonto mit dem Namen \lstinline{admin}.

\subsubsection{Antwort (bei Erfolg)}

\lstinline{Content-Type: application/json}
\begin{lstlisting}
    {
        "success": true, 
        "message": "ACP token for this email is valid"
    }
\end{lstlisting}

\subsubsection{Antwort (bei ungültigen Token)}

\lstinline{Content-Type: application/json}
\begin{lstlisting}
    {
        "success": true, 
        "message": "Could not validate ACP token for username 'admin'"
    }
\end{lstlisting}
