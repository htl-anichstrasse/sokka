\subsection{/acp/acpuser/create}

\begin{lstlisting}
    POST /acp/acpuser/create
\end{lstlisting}

\textbf{Beschreibung:} \\
Erstellt ein neues ACP-Nutzerkonto. Wird genutzt, um weitere ACP-Nutzerkonten im ACP zu erstellen.

\textbf{Authorization-Typ:} \\
ACP

\subsubsection{Payload}

\lstinline{name: string} (benötigt) \\
$\rightarrow$ Der gewünschte Name des neu anzulegenden ACP-Nutzerkontos

\lstinline{password: string} (benötigt) \\
$\rightarrow$ Das gewünschte Passwort des neu anzulegenden ACP-Nutzerkontos

\subsubsection{Beispiel}

\begin{lstlisting}
    POST /acp/acpuser/create
    {
        "name": "joshua",
        "password": "sicheresPasswort"
    }
\end{lstlisting}

Erstellt ein ACP-Nutzerkonto mit dem Namen \glqq joshua\grqq\space und dem Passwort \glqq sicheresPasswort\grqq.

\subsubsection{Antwort (bei Erfolg)}

\lstinline{Content-Type: application/json}
\begin{lstlisting}
    {
        "success": true, 
        "message": "Successfully created ACP user with username 'joshua'"
    }
\end{lstlisting}