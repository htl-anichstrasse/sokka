\subsection{/order/get}

\begin{lstlisting}
    GET /order/get
\end{lstlisting}

\textbf{Beschreibung:} \\
Gibt alle Bestellungen des angemeldeten Nutzerkontos zurück, oder nur jene, welche an einem angegebenen Datum getätigt wurden. Wird genutzt, um getätigte Bestellungen im Client aufzulisten.

\textbf{Authorization-Typ:} \\
App

\subsubsection{Parameter}

\lstinline{date: string} (optional)
$\rightarrow$ Das Datum der abzurufenden Bestellungen.

\subsubsection{Beispiel}

\begin{lstlisting}
    GET /product/get?date=2021-03-03
\end{lstlisting}

Gibt alle Bestellungen des angemeldeten Nutzerkontos zurück, welche am 3. März 2021 getätigt wurden.

\subsubsection{Antwort (bei Erfolg)}

\lstinline{Content-Type: application/json}
\begin{lstlisting}
    {
        "success": true, 
        "orders": [
            {
                "id": 60,
                "user_id": 26,
                "timestamp": "2021-03-01T08:29:34.000Z"
                "state": "VALID",
                "rebate": 10,
                "total": 6,
                "menuOrders": [
                    "menu_id": 1,
                    "quantity": 1,
                    "menu": <Menu Object>
                ],
                "productOrders": [
                    "product_id": 1,
                    "quantity": 1,
                    "product": <Product Object>
                ]
            }
        ]
    }
\end{lstlisting}
