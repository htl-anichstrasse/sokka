\subsection{/verify}
\label{verify}

\begin{lstlisting}
    GET /verify
\end{lstlisting}

\textbf{Beschreibung:} \\
Überprüft einen E-Mail-Verifizierungs-Token auf Validität. Diesen erhalten Nutzer nach der Erstellung eines Nutzerkontos via E-Mail.

\textbf{Authorization-Typ:} \\
Keine

\subsubsection{Parameter}

\lstinline{id: string} (benötigt) \\
$\rightarrow$ Der Verifizierungs-Token

\subsubsection{Beispiel}

\begin{lstlisting}
    GET /verify?id=9n6g7M7ow4qCNXW2nmkpYEavx8M32c%2BRGk1gyYq2ymJ8MRVfx98Bb0bRqt9VrtC40YcqKWpRk86FAShSH4zhQ%3D%3D
\end{lstlisting}

Prüft den angegebenen Verifizierungstoken

\subsubsection{Antwort (bei Erfolg)}

\lstinline{Content-Type: application/json}
\begin{lstlisting}
    {
        "success": true, 
        "message": "Successfully verified user"
    }
\end{lstlisting}

\subsubsection{Antwort (bei ungültigem Token)}

\lstinline{Content-Type: application/json}
\begin{lstlisting}
    {
        "success": false, 
        "message": "Invalid token '9n6g7M7ow4qCNXW2nmkpYEavx8M32c%2BRGk1gyYq2ymJ8MRVfx98Bb0bRqt9VrtC40YcqKWpRk86FAShSH4zhQ%3D%3D'" 
    }
\end{lstlisting}