\subsection{/user/login}
\label{appauth}

\begin{lstlisting}
    POST /user/login
\end{lstlisting}

\textbf{Beschreibung:} \\
Führt eine Anmeldung bei einem existenten Nutzerkonto durch. Wird genutzt, um neue Anmeldungen für Nutzerkonten im Client zu ermöglichen.

\textbf{Authorization-Typ:} \\
Keine

\subsubsection{Payload}

\lstinline{email: string} (benötigt) \\
$\rightarrow$ Die E-Mail-Adresse des Nutzerkontos, bei dem die Anmeldung durchzuführen ist

\lstinline{password: string} (benötigt) \\
$\rightarrow$ Das Passwort des Nutzerkontos, bei dem die Anmeldung durchzuführen ist

\subsubsection{Beispiel}

\begin{lstlisting}
    POST /user/login
    {
        "email": "meineEmail@sokka.me",
        "password": "test123"
    }
\end{lstlisting}

Führt eine Anmeldung beim Nutzerkonto mit der E-Mail-Adresse \lstinline{meineEmail@sokka.me} und dem Passwort \lstinline{test123} durch.

\subsubsection{Antwort (bei Erfolg)}

\lstinline{Content-Type: application/json}
\begin{lstlisting}
    {
        "success": true, 
        "message": "Credentials validated",
        "token": "eYiKOYqtihKgVknurEq1luc3mhawIxAo"
    }
\end{lstlisting}

Bei Erfolg wird ein Session-Token generiert und zurückgegeben.