\subsection{/acp/acpuser/get}

\begin{lstlisting}
    GET /acp/acpuser/get
\end{lstlisting}

\textbf{Beschreibung:} \\
Gibt alle im System gespeicherten ACP-Nutzerkonten zurück. Wird genutzt, um eine Auflistung von ACP-Nutzerkonten im ACP zu erstellen.

\textbf{Authorization-Typ:} \\
ACP

\subsubsection{Parameter}
Keine

\subsubsection{Antwort (bei Erfolg)}

\lstinline{Content-Type: application/json}
\begin{lstlisting}
    {
        "success": true, 
        "users": [
            {
                "name": "joshua",
                "timestamp": "2021-03-02T18:22:02.043Z"
            },
            {
                "name": "nicolaus",
                "timestamp": "2021-03-02T18:23:08.890Z"
            }
        ]
    }
\end{lstlisting}

Wobei \lstinline{timestamp} der Timestamp der Erstellung des Nutzerkontos im ISO-8601-Format ist.