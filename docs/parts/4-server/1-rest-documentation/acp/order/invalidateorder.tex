\subsection{/acp/order/invalidate}

\begin{lstlisting}
    POST /acp/order/invalidate
\end{lstlisting}

\textbf{Beschreibung:} \\
Invalidiert eine Bestellung. Wird genutzt, um eingelöste Bestellungen bzw. QR-Codes an der Kasse zu invalidieren. Achtung: Diese Route prüft nicht, ob die Bestellung einlösbar ist. Dafür ist \nameref{validateOrder} zu nutzen.

\textbf{Authorization-Typ:} \\
ACP

\subsubsection{Payload}

\lstinline{order: string} (benötigt) \\
$\rightarrow$ Die zu invalidierende Bestellung im Format \glqq userId:orderId\grqq

\subsubsection{Beispiel}

\begin{lstlisting}
    POST /acp/menu/invalidate
    {
        "order": "25:10",
    }
\end{lstlisting}

Invalidiert die Bestellung mit der ID \lstinline{10}.

\subsubsection{Antwort (bei Erfolg)}

\lstinline{Content-Type: application/json}
\begin{lstlisting}
    {
        "success": true, 
        "message": "Successfully invalidated order"
    }
\end{lstlisting}