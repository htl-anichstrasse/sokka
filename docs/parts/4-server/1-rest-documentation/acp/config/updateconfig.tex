\subsection{/acp/config/update}

\begin{lstlisting}
    POST /acp/config/update
\end{lstlisting}

\textbf{Beschreibung:} \\
Aktualisiert einen Config-Eintrag, insbesondere dessen Wert. Wird genutzt, um im ACP die Werte der Config-Einträge änderbar zu machen.

\textbf{Authorization-Typ:} \\
ACP

\subsubsection{Payload}

\lstinline{key: string} (benötigt) \\
$\rightarrow$ Der Key des zu aktualisierenden Config-Eintrags

\lstinline{type: string} (optional) \\
$\rightarrow$ Der Typ des zu aktualisierenden Config-Eintrags

\lstinline{friendlyName: string} (optional) \\
$\rightarrow$ Der Anzeigename des zu aktualisierenden Config-Eintrags

\lstinline{value: string} (optional) \\
$\rightarrow$ Der Wert des zu aktualisierenden Config-Eintrags

Wobei \lstinline{type} nur \lstinline{INTEGER}, \lstinline{STRING} oder \lstinline{TIME} sein kann.

\subsubsection{Beispiel}

\begin{lstlisting}
    POST /acp/config/update
    {
        "key": "closingTime",
        "value": "17:00"
    }
\end{lstlisting}

Setzt den Zeitwert vom Config-Eintrag mit dem Key \lstinline{closingTime} auf den Wert \lstinline{17:00}.

\subsubsection{Antwort (bei Erfolg)}

\lstinline{Content-Type: application/json}
\begin{lstlisting}
    {
        "success": true, 
        "message": "Successfully updated config entry"
    }
\end{lstlisting}