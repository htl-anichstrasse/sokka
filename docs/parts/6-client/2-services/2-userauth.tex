\section{User-Authentication}



\subsection{Registrieren eines neuen Users}

Für die Registrierung eines neuen Nutzers müssen seine E-Mail-Adresse inklusive seines 
gewählten Passwords an die API gesendet werden.\\
Dies geschieht über die Form im Signup-Screen.

\begin{lstlisting}
Future<String> signUpUser(final String email, final String password) async {
    return await this._networkWrapper.post(
        SIGNUP_ROUTE,
        headers: {
            'Content-Type': 'application/json',
        },
        body: {
            'email': email,
            'password': password,
            "tos": true,
            "privacypolicy": true,
        },
    ).then((response) {
        return response['token'];
    });
}
\end{lstlisting}

Ist der Request erfolgreich, so wird ein neuer Nutzer in der Datenbank erzeugt. Als Antwort wird der
Token für die damit generierte User-Session übermittelt, welcher in weiterer Folge im CookieStorage
abgespeichert wird.

\subsection{Anmelden eines vorhandenen Users}

\begin{lstlisting}
Future<String> loginUser(final String email, final String password) async {
    return await this._networkWrapper.post(
        LOGIN_ROUTE,
        headers: {
            'Content-Type': 'application/json',
        },
        body: {
            'email': email,
            'password': password,
        }
    ).then((response) {
        return response['token'];
    });
}
\end{lstlisting}

\subsection{Abmelden eines Users}

Wenn sich ein User manuell in der App abmeldet wird ein Request an \lstinline{/user/logout} mit
dem gespeicherten Session-Token und der E-Mail des Nutzers gesendet.

Jener Session-Token wird in weiterer Folge ungültig gemacht und aus dem Cookie-Storage der App
gelöscht, wodruch der Nutzer beim nächsten App-Start wieder zum Login-Screen weitergeleitet wird.

\begin{lstlisting}
Future<void> logoutUser(final String sessionToken, final String email) async {
    await this._networkWrapper.post(
        LOGOUT_ROUTE,
        headers: {
            'Content-Type': 'application/json',
        },
        body: {
            'token': sessionToken,
            'email': email
        },
    );
}
\end{lstlisting}

\subsection{Validieren einer User-Session}

Damit sich ein Nutzer der Sokka-App nicht bei jedem Start der App mit seinen Logindaten neu
anmelden muss, werden die E-Mail-Adresse und der Session-Token nach einer erfolgreichen Anmeldung
im CookieStorage abgespeichert.

Mit jenen Daten im Speicher kann nun bei jedem weiteren App-Start ein Request an die
\lstinline{/user/validate}-Route mit Session-Token und E-Mail zur Validierung der User-Session
gesendet werden.\\
Wenn der Session-Token nachwievor gültig ist wird der User automatisch zum Home-Screen weitergeleitet.
Ist die Session abgelaufen erscheint der Login- bzw. Signup-Screen.

\begin{lstlisting}
Future<bool> validateSessionToken(final String sessionToken, final String email) async {
    return await this._networkWrapper.post(
        VALIDATE_ROUTE,
        headers: {
            'Content-Type': 'application/json',
        },
        body: {
            'token': sessionToken,
            'email': email
        },
    ).then((response) {
        return response['success'];
    });
}
\end{lstlisting}

\subsubsection{Wrapper-Methode zur Session-Validierung}

Damit der Check der Validität des gespeicherten Session-Tokens einfach
bei Start der App ausgeführt werden kann gibt es folgende Wrapper-Methode,
die automatisch benötigte Werte aus dem Storage nimmt und einen entsprechenden
Request an die Sokka-API sendet.

\begin{lstlisting}
Future<bool> validateSession() async {
    String email = await this._cookieStorage.getEmail();
    String token = await this._cookieStorage.getSessionToken();

    return await this.validateSessionToken(token, email);
}
\end{lstlisting}