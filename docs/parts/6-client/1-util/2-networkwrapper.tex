\section{Network-Wrapper}

Um in einer Flutter-App Requests sowie Responses über das \textit{HTTP}-Protokoll versenden und empfangen zu können
muss das von Dart bereitgestellte \textit{http}-Package zu den Dependencies im \lstinline{pubspec.yaml}-File hinzugefügt
werden.

\begin{lstlisting}
dependencies:
  http: ^0.12.2
\end{lstlisting}

Mithilfe dieses Packages können nun \textit{GET}- und \textit{POST}-Requests an eine bestimmte URL, in unserem Fall
an die Sokka-API gesendet werden, wobei die erhaltene Antwort per \lstinline{jsonDecode()}-Funktion in ein lesbares
JSON-Format umgewandelt werden muss.

\begin{lstlisting}
import 'dart:convert';
import 'package:http/http.dart' as http;

void main (List<String> arguments) async {
    final response = await http.post(
        'https://api.sokka.me/user/login',
        headers: {
            ...
        },
        body: jsonEncode({
            email: 'email',
            password: 'password'
        }),
    );
}
\end{lstlisting}

Um das Versenden und Verarbeiten solcher HTTP-Requests möglichst einfach zu gestalten bietet jene
Klasse zwei Wrapper-Funktionen für GET- und POST-Requests an.

Jene Funktionen erhalten die \textit{Ziel-URL} und entsprechendend \textit{Request-Headers} und \textit{-Body} in Form einer HashMap
als Argumente, welche direkt per \lstinline{jsonEncode()} umgewandelt werden.

Tritt bei einem durchgeführten Request ein Fehler auf - ein Statuscode der Antwort unter 200 bzw. über 400 oder ein
leeres JSON-Objekt - wirft die Funktion eine Exception mit entsprechendem HTTP-Code.

Erfolgt der Request fehlerlos so wird dessen Body per \lstinline{jsonDecode()} umgewandelt und retourniert.

\subsubsection{NetworkWrapper.GET}

\begin{lstlisting}
Future<dynamic> get(final String url, { final Map<String, String> headers }) async {
    return await http.get(
        url,
        headers: headers,
    ).then((http.Response response) {
        if (response.statusCode < 200 || response.statusCode > 400 || json == null) {
            throw new Exception("Error fetching data from $url with ${response.statusCode}");
        }
        return this._jsonDecoder.convert(response.body);
    });
}
\end{lstlisting}

\subsubsection{NetworkWrapper.POST}

\begin{lstlisting}
Future<dynamic> post(final String url, { final Map<String, String> headers, body, encoding }) async {
    return await http.post(
        url,
        headers: headers,
        body: this._jsonEncoder.convert(body),
        encoding: encoding
    ).then((http.Response response) {
        if (response.statusCode < 200 || response.statusCode > 400 || json == null) {
            throw new Exception("Error fetching data from $url");
        }
        return this._jsonDecoder.convert(response.body);
    });
}
\end{lstlisting}