\section{Controllers}

Mithilfe der Controller-Klassen können benötigte Werte oder Informationen global in der App
gespeichert und dynamisch geändert und angezeigt werden, beispielsweise die Produkte, die sich 
im Warenkorb des Nutzers befinden.

\subsubsection{Models für Controller}

\paragraph{Menu}

Im Menu-Model werden alle relevanten Daten zu einem Menü abgespeichert.

\begin{tabular}{l l r l}
    Integer & \_menuID & $\rightarrow$ & ID des Menüs\\
    String & \_name & $\rightarrow$ & Name des Menüs\\
    List$<$MenuEntry$>$ & \_entries & $\rightarrow$ & die Liste mit allen Menü-Einträgen\\
    Double & \_price & $\rightarrow$ & Preis des Menüs\\
    NetworkImage & \_image & $\rightarrow$ & Bild des Menüs\\
    Boolean & \_isHidden & $\rightarrow$ & Status, ob das Menü im View angezeigt werden soll
\end{tabular}

\paragraph{MenuEntry}

Verknüpft alle Produkte mit dem Menü, in dem sie enthalten sind.

\begin{tabular}{l l r l}
    Integer & \_titleID & $\rightarrow$ & Name des Produkts im Menü\\
    Integer & \_menuID & $\rightarrow$ & ID des Menüs\\
    Product & \_product & $\rightarrow$ & das im Menü enthaltene Produkt\\
\end{tabular}
 
\paragraph{Product}

Im Product-Model werden alle relevanten Daten zu einem Produkt abgespeichert.

\begin{tabular}{l l r l}
    Integer & \_productID & $\rightarrow$ & ID des Produkts\\
    String & \_name & $\rightarrow$ & Name des Produkts\\
    Double & \_price & $\rightarrow$ & Preis des Produkts\\
    NetworkImage & \_image & $\rightarrow$ & Bild des Produkts\\
    Boolean & \_isHidden & $\rightarrow$ & Status, ob das Produkt im View angezeigt werden soll
\end{tabular}

\paragraph{Order}

Im Order-Model werden alle relevanten Daten zu einer Bestellung abgespeichert.

\begin{tabular}{l l r l}
    Integer & \_orderID & $\rightarrow$ & ID der Bestellung\\
    Integer & \_userID & $\rightarrow$ & ID des Nutzers, der die Bestellung getätigt hat\\
    String & \_timestamp & $\rightarrow$ & Zeitpunkt, an dem die Bestellung erstellt worden ist\\
    Integer & \_rebate & $\rightarrow$ & Rabatt, der vom Gesamtpreis der Bestellung abgezogen wird\\
    Double & \_subTotal & $\rightarrow$ & Preis der Bestellung, ohne den Rabatt zu berücksichtigen\\
    Boolean & \_total & $\rightarrow$ & Preis der Bestellung mit abgezogenem Rabatt\\
    Map$<$String, int$>$ & \_menuOrders & $\rightarrow$ & Map mit allen bestellten Menüs und ihrer Anzahl\\
    Map$<$String, int$>$ & \_productOrders & $\rightarrow$ & Map mit allen bestellten Produkten und ihrer Anzahl\\
    String & \_qrData & $\rightarrow$ & String, aus dem der QR-Code der Bestellung generiert wird\\
\end{tabular}

