\section{Controllers}
\label{controllers}

Mithilfe der Controller-Klassen können benötigte Werte oder Informationen eines bestimmten Views
global in der App gespeichert, dynamisch geändert und angezeigt werden, beispielsweise die Produkte,
die sich im Warenkorb des Nutzers befinden.

\subsection{Models für Controller}

\paragraph{Menu}
\label{menucontroller}

Im Menu-Model werden alle relevanten Daten zu einem Menü abgespeichert.

\begin{tabular}{l l l}
    Integer & \_menuID & ID des Menüs\\
    String & \_name & Name des Menüs\\
    List$<$MenuEntry$>$ & \_entries & die Liste mit allen Menü-Einträgen\\
    Double & \_price & Preis des Menüs\\
    NetworkImage & \_image & Bild des Menüs\\
    Boolean & \_isHidden & Status, ob das Menü im View angezeigt werden soll
\end{tabular}

\paragraph{MenuEntry}

Verknüpft alle Produkte mit dem Menü, in dem sie enthalten sind.

\begin{tabular}{l l l}
    Integer & \_titleID & Name des Produkts im Menü\\
    Integer & \_menuID & ID des Menüs\\
    Product & \_product & das im Menü enthaltene Produkt\\
\end{tabular}
 
\paragraph{Product}
\label{productcontroller}

Im Product-Model werden alle relevanten Daten zu einem Produkt abgespeichert.

\begin{tabular}{l l l}
    Integer & \_productID & ID des Produkts\\
    String & \_name & Name des Produkts\\
    Double & \_price & Preis des Produkts\\
    NetworkImage & \_image & Bild des Produkts\\
    Boolean & \_isHidden & Status, ob das Produkt im View angezeigt werden soll
\end{tabular}
\pagebreak
\paragraph{Order}

Im Order-Model werden alle relevanten Daten zu einer Bestellung abgespeichert.

\begin{tabular}{l l l}
    Integer & \_orderID & ID der Bestellung\\
    Integer & \_userID & ID des Nutzers, der die Bestellung getätigt hat\\
    String & \_timestamp & Zeitpunkt, an dem die Bestellung erstellt worden ist\\
    Integer & \_rebate & Rabatt, der vom Gesamtpreis der Bestellung abgezogen wird\\
    Double & \_subTotal & Preis der Bestellung, ohne den Rabatt zu berücksichtigen\\
    Boolean & \_total & Preis der Bestellung mit abgezogenem Rabatt\\
    Map$<$String, int$>$ & \_menuOrders & Map mit allen bestellten Menüs und ihrer Anzahl\\
    Map$<$String, int$>$ & \_productOrders & Map mit allen bestellten Produkten und ihrer Anzahl\\
    String & \_qrData & String, aus dem der QR-Code der Bestellung generiert wird\\
\end{tabular}

\subsection{Menu-Controller}

Der Menu-Controller speichert alle zum aktuellen Zeitpunkt verfügbaren Menüs in einer internen Liste,
die zu App-Start per API-Call befüllt wird, ab.\\
Ebenso wie der \textit{CookieStorage} ist auch der Menu-Controller - sowie alle weiteren Controller -
als Singleton implementiert.

\subsection{Product-Controller}

Im Product-Controller werden alle zum Kauf verfügbaren Produkte abgespeichert. Auch dieser wird
bei App-Start durch einen Aufruf der API initialisiert.

\subsection{ShoppingBasket-Controller}

In der Liste des ShoppingBasket-Controllers werden all jene Menüs und Produkte zwischengespeichert, die 
vom Nutzer zu seinem Warenkorb hinzugefügt worden sind.\\
Im Gegensatz zu den anderen Controllern ist dieser beim Start der App leer und wird erst durch
die Interaktion des Users befüllt.

\subsection{Order-Controller}

Der Order-Controller verwaltet alle vom Nutzer getätigten Bestellungen und wird gleich wie der
Menu- und Product-Controller zu Beginn über die Sokka-API befüllt.
