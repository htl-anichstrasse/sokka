\section{Dart}

\subsection{Was ist Dart?}

\textit{Dart} ist eine höhere, für den client-optimierte Programmiersprache
zur Entwicklung von Apps auf mehreren Plattformen.\\
Sie wird von \textit{Google} entwickelt und wird hauptsächlich zum Entwickeln von
Servern, (nativen) Mobile-, Desktop- oder Web-Apps verwendet.

Wie viele andere Hochsprachen baut Dart strikt auf dem Konzept der Objekt-Orientierung auf
und ist damit eine klassenbasierte Programmiersprache mit eingebautem, automatischem
Garbage-Collector.

Dart ist eine kompilierte Sprache und kann sowohl in Maschinencode als auch in \\
Vanilla-JavaScript kompiliert werden, wodurch plattformunabhängige Entwicklung ermöglicht 
wird.

\subsection{Syntax von Dart}

Die Syntax von Dart sieht auf den ersten Blick aus, wie eine Mischung aus \textit{Java}\\
und \text{JavaScript}.

Ein \textit{Hello-World}-Programm würde in Dart also folgendermaßen aussehen:
\begin{lstlisting}
void main(List<String> arguments) {
    print('Hello, World!');
}
\end{lstlisting}

\subsubsection{Variablen}

Die Deklaration von Variablen funktioniert gleich wie anderen Hochsprachen,
wobei Primitiv- und Referenzdatentypen (int, String, Integer, ...) aus Sprachen wie Java zur Verwendung
kommen.\\
Alternativ kann auf das Schlüsselwort \textit{var} aus JavaScript zurückgegriffen werden. Dieses
ist vergleichbar mit dem \textit{Object}-Datentyp aus Java und ''errät'' den benötigten
Typ der entsprechenden Variable.

\begin{lstlisting}
// Direktes Anlegen eines Integers mit dem Wert 5.
int x = 5;

// Anlegen eines Integers mit dem Wert 7 mithilfe von ''var''.
var y = 7;
\end{lstlisting}

Nach Dart-Konventionen ist es allerdings üblich, die Angabe des Datentyps für Variablen 
im lokalen Scope, bspw. innerhalb einer Funktion durch das \textit{var}-Keyword zu ersetzen.

Für \textit{finale} Variablen kann sowohl auf dezidierten Datentyp als auch auf \textit{var} 
verzichtet werden.

\begin{lstlisting}
final x = 10;
\end{lstlisting}

\subsubsection{Verzweigungen und Schleifen}

Für die Kontrolle des Ablaufs eines Programms werden wie gewohnt \textit{if-else-Statements} und Schleifen
wie \textit{for / for in / for each} oder \textit{while} verwendet.\\
Die Syntax jener Kontrollstrukturen ist mit der von JavaScript quasi ident.

\begin{lstlisting}
// Verzweigung mit zwei Bedingungen
if (condition 1) {
    ...
} else if (condition 2) {
    ...
} else {
    ...
}
\end{lstlisting}

\begin{lstlisting}
final list = [1, 2, 3, 4, 5];

// Standard for-Schleife
for (var i; i < list.length; i++) {
    ...
}

// for-in-Schleife
for (var value in list) {
    ...
}

// forEach-Schleife mit Callback-Funktion
list.forEach((value) => {
    ...
})
\end{lstlisting}

\begin{lstlisting}
//while-Schleife
while(!condition) {
    ...
}
\end{lstlisting}

\subsubsection{Funktionen}

Funktionen in Dart werden auf dieselbe Weise wie in Java deklariert. Die Funktionssignatur 
bzw. der Funtkionskopf benötigt den Rückgabewert der Funktion, den Namen und die Parameterliste.

\begin{lstlisting}
// Funktion ohne Rueckgabewert
void sayHi() {
    print('Hi!');
}

int multiply(final int x, final int y) => x * y;
\end{lstlisting}

Dart bietet in Bezug auf das Übergeben von Parametern an eine Funktion mehrere Mögichkeiten:
\begin{itemize}
    \item \textit{positioned, optional parameters}
    \item \textit{named, optional parameters}
\end{itemize}

Sogenannte \textit{positionierte und optionale Parameter} werden in der Argumentliste einer Funktion
mithilfe eckiger Klammern gekennzeichnet und bewirken, wie der Name bereits verrät, dass jene Parameter
optional und in ihrer Position fixiert sind.

\begin{lstlisting}
void sendRequest(final String url, final String message, [ final int port = 80 ]) {
    ...
}
\end{lstlisting}

Die Funktion \textit{getHttpUrl()} benötigt neben dem Server und Pfad auch eine Portnummer, welche
hier als positionierter, optionaler Parameter mit einem Standartwert von 80 festgelegt wird.
Wird beim Aufruf dieser Funktion der Port-Integer weggelassen, so wird dieser immer auf 80 gesetzt.

\begin{lstlisting}
// Aufruf mit Port 8080
getHttpUrl('sokka.me', 'Sokka is awesome!', 8080);

// Aufruf mit Standartport 80
getHttpUrl('sokka.me', 'Sokka is awesome!');
\end{lstlisting}

Neben positionierten Parametern gibt es noch ''benannte'', sogenannte \textit{named} Parameter, 
die mit geschwungenen Klammern in der Parameterliste deklariert werden.\\

\begin{lstlisting}
sendRequest(final String url, { final String message, final int port }) {
    ...
}
\end{lstlisting}

Neben einer URL benötigt die obige Funktion auch eine Nachricht und eine Port-Nummer,
um einen gültigen Request zu versenden.\\
Aufgrund ihrer Deklaration innerhalb der geschwungen Klammern
werden sie als benannte Parameter interpretiert und werden wie folgt übergeben.

\begin{lstlisting}
sendRequest('api.sokka.me', message: 'Sokka is awesome!', port: 80);
\end{lstlisting}

\subsubsection{Klassen}

Ähnlich zu Hochsprachen wie Java oder C\# können auch in Dart einfach Klassen
erstellt werden, wie nachfolgend zu sehen ist.

\begin{lstlisting}[language=JavaScript]
class Point {
    int _x;
    int _y;
    
    Point(this._x, this._y);
}
\end{lstlisting}

Wie auch in JavaScript oder Python gibt es in Dart keine Access-Modifier um den Zugriff
auf klasseninterne Felder oder Methoden zu regulieren.\\
Felder und Funktionen sind standardmäßig \textbf{public} und können lediglich durch
einen Unterstrich zu Beginn des Feld- bzw. Funktionsnamen als \textbf{private} gekennzeichnet
werden.

Desweiteren stehen in Dart - ebenso wie in C\# - die Schlüsselwörter \textbf{get} und 
\textbf{set} zur Verfügung, mit deren Verwendung Funktionen als dezidierte Getter- bzw.
Setter-Funktion für eine Membervariable definiert werden können.

\begin{lstlisting}[language=JavaScript]
class Foo {
    String _bar;

    String get getBar => this._bar;

    set setBar(final String bar) => this._bar = bar;
}
\end{lstlisting}

In diesem Beispiel werden \textit{getBar} und \textit{setBar} als dezidierte Getter- und\\
Setter-Funktion für das Feld \textbf{\_bar} festgelegt.\\
Auffällig ist ebenso die JavaScript-ähnliche Arrow-Syntax, die ein Return-Statement
zur Gänze ersetzen kann.
So kann ein \textit{return}, wie es in Java geschrieben wird, durch ein Arrow-Statement wie folgt
ersetzt werden.

\begin{lstlisting}[language=JavaScript]
// Java
public String getName() {
    return this.name;
}

// Dart
String get getName => this._name;
\end{lstlisting}