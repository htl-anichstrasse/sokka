\section{JavaScript}
\label{javascript}

\textit{JavaScript} ist eine 1995 erschienene funktionale Skriptsprache mit dynamischem Typensystem, welche ursprünglich für die Erstellung von Websites mit dynamischen Inhalten erstellt wurde. Heutzutage ist JavaScript überall und hält den ersten Platz der meistverwendeten Programmiersprachen. \cite{technostacks2021}

In Sokka findet JavaScript nur in Verwendung mit \nameref{gulpjs} und als kompilierte Version von \nameref{typescript} und \nameref{dart} Anwendung.

\subsection{Kritik an JavaScript}

Obwohl JavaScript heutzutage eine der am weitesten verbreiteten Programmiersprachen ist und von vielen anerkannten Firmen aktiv verwendet wird, streuben sich einige Entwickler dennoch gegen die Nutzung von JavaScript.~\cite{keenetheng2016}

Die Begründung liegt oftmals bei den eigenartigen Verhaltensweisen der Sprache, welche man aus anderen Programmiersprachen nicht gewohnt ist. Listen wie \textit{wtfjs} von \textit{Denys Dovhan} zeigen Beispiele, wie komisch JavaScript wirklich sein kann. Einige Auszüge davon (siehe denysdovhan/wtfjs auf GitHub für alle Einträge): 

\begin{code}[htp]
    \begin{center}
        \includegraphics[width=1\textwidth]{images/JavaScript/fail.png}
        \vspace{-25pt}
        \caption{\glqq fail\grqq\space in JavaScript; diese Art von JavaScript ist auch bekannt als \textit{JSFuck}}
    \end{center}
\end{code}

\begin{code}[htp]
    \begin{center}
        \includegraphics[width=1\textwidth]{images/JavaScript/coercing.png}
        \vspace{-25pt}
        \caption{Lustige Mathematik: Die Typenumwandlung (Coercing) in JavaScript ...}
    \end{center}
\end{code}